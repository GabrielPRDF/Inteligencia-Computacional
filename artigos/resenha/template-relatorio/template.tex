% =======================================================================
% =                                                                     =
% =======================================================================
% -----------------------------------------------------------------------
% - Author:     Chaua Queirolo                                          -
% - Version:    001                                                     -
% -----------------------------------------------------------------------
\documentclass[a4paper,11pt]{article}    

% =======================================================================
% PACKAGES
% =======================================================================

% Language support
\usepackage[brazil]{babel}
\usepackage[utf8]{inputenc}
\usepackage[T1]{fontenc}
\usepackage{ae,aecompl}

% Configuration
\usepackage{url}
\usepackage{enumerate}
\usepackage{color}
\usepackage[svgnames,table]{xcolor}
\usepackage[margin=2cm,includefoot]{geometry}

% Tabular
\usepackage{multirow}
\usepackage{multicol}

% Images
\usepackage{graphicx}
\usepackage[scriptsize]{subfigure}
\usepackage{epstopdf}
\usepackage{float}% http://ctan.org/pkg/{multicol,lipsum,graphicx,float}

% Math
\usepackage{mdwtab}	% bug rowcolor
\usepackage{amssymb}
\usepackage{amsmath}
\usepackage{footnote}

% References
\usepackage[sort,nocompress]{cite}

% =======================================================================
% VARIABLES
% =======================================================================

% Space between the lines in a table
\renewcommand{\arraystretch}{1.3}

% Define a new column type
\newcolumntype{x}[1]{>{\raggedright\hspace{0pt}}p{#1}}%

% Controle das Margens
\sloppy
\tolerance=9999999

% Espaço entre colunas
\setlength{\columnsep}{.9cm}


% Configuration
\usepackage{lipsum}
\usepackage{blindtext}

% =======================================================================
% HEADER
% =======================================================================

\title{Resenha}
\author{Gabriel Pinto Ribeiro da Fonseca\\E-mail: {\tt gabriel-prdf@hotmail.com}}
\date{}

\newenvironment{Figure}
  {\par\medskip\noindent\minipage{\linewidth}}
    {\endminipage\par\medskip}

% =======================================================================
% DOCUMENT
% =======================================================================
\begin{document}

\maketitle


\begin{multicols}{2}


Freitas\cite{ref:freitas2014} informa que a adulteração de leite por adição de soro de queijo vem ocorrendo devido a seu baixo valor comercial, como este recurso não tem muita possibilidade para sua utilização e seu devido tratamento ser caro. É estimado que para cada queijo produzido se obtém de 9 a 12 litros de soro de queijo.
A adição deste soro é detectada pelo caseinomacropeptídeo(CMP), este é um fragmento hidrofílico da K-cascína.  Esta substancia é detectada através de métodos cromatográficos, eletroforéticos e espectrofotométricos. Sendo todos estes métodos caros tanto em equipamentos como recursos.
Neste artigo Freitas\cite{ref:freitas2014}, busca elaborar um novo método de análise do leite, para classificar o normal e o adulterado, de forma mais eficiente, utilizando de redes neurais.
É justificada o uso de redes neurais através de características que esta metodologia possui, como utilizar variáveis de entrada e saída e assim se obter um resultado, aprender com padrões já analisados.

Para a criação do método foi coletada 95 amostras de leite sem adulteração, 72 amostras com adulteração, seguindo a porcentagem de 1, 5, 10 e 20 com 18 amostras cada. Na análise destes conteúdos utilizaram de um analisador de leite por ultrassom (MASTER COMPLETE-AKSO). As variáveis levadas em critério foram: teor de gordura, extrato seco desengordurado, densidade, proteína, lactose, sais minerais, ponto de congelamento, condutividade, PH e temperatura.
Os soros foram coletados na produção de queijo Minas e Muçarela, após suas coletas, filtrou-se para retirar partículas grosseiras vindas do queijo.
Para melhorar a elaboração da arquitetura da rede neural, distribuíram os 167 ensaios de leite aleatoriamente para as camadas de treinamento (60\%), validação (20\%) e teste (20\%).
O software SNN foi utilizado para a configuração da rede, e escolhido dois tipos, redes de função radial e multilayer perceptron. O critério de seleção usado foi o EQM, erro quadrático médio.
Após a inserção das variáveis, configuraram duas maneiras para a saída, a primeira era de (0) para leite normal e (1) para adulterado e a outra, (0,1) leite normal e (1,0), para adulterado.
Os algoritmos usados neste artigo foram, K-means, K-nearest e o método pseudo-inversa. K-means foi responsável por definir os centros da rede, k-nearest a largura dos campos receptivos e o pseudo-inverso os pesos da camada de saída. Na multilayer perceptron, o algoritimo utilizado para o treinamento foi o retropropagação.
As redes com o menos erro de classificação, apresentaram ser as de apenas dois neurônios, com os melhores resultados comparando com 28.017 redes testadas.
A melhor rede obtida é a de função de base radial, em que esta possuía 10 neurônios na camada de entrada, 40 na camada oculta e apenas duas na saída. O resultado obtido com as simulações teve um erro de erro de 4.8\% para os leites alterados e 0\% para o leite normal.
Conclusão
Com os resultados obtidos, é mostrado que a rede neural tem um aproveitamento superior ao das metodologias empregadas atualmente, como a metodologia da estatística multivariada, que possui um erro que de 36\% a 30\%, enquanto o erro atingido pelo trabalho e inferior a 5\%.


\bibliographystyle{plain}
\bibliography{referencias}

\end{multicols}
\end{document}

